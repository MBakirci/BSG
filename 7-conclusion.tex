% Real-Time Strategy games have been around for a very long time. However, the elements of RTS games have not changed much over the last years. Gathering resources, building units and buildings and scouting the environment to ultimately defeat the opponent have been the main goals of most RTS games. All these tasks combined can be very complex and has been proven an interesting field of research for Artificial Intelligence.

This paper has presented why RTS games are currently seen as a new horizon for AI research and has delved into a selection of state-of-the-art AI algorithms, hopefully providing a good starting point to those who wish to learn more about this research field.


% Misschien is dit juist wel allemaal nice voor in de intro/voor de research question
% A lot of research has already been done on certain topics in RTS games. Most of that research is focused on one or more tasks like planning, producing units or micromanagement. Often, different research on the same task proposes a different approach, which in turn delivers different performance. Because there already has been a lot of research on a lot of different tasks, a selection has been made from all of the research papers that have been surveyed. The selection has been made based on a variety of promising techniques applied to different tasks in RTS games, to cover a lot of topics. The purpose of this literature survey is to give the reader a notion of different Artificial Intelligence techniques that can be used to train agents that execute specific tasks in Real-Time Strategy games. 

Of all the algorithm reviews, the first technique is the learning of macromanagement based on live-action replays. This technique uses a neural network which is trained to predict the next build  on a given game state. The advantage of this, is that training times can be improved enormously. However, the performance is always related to the quality of the training data. 

The second technique is the use of a Genetic Algorithm to optimize building placement. The Genetic Algorithm tries to find the optimal building placement which can turn an attack on the base from a major loss into a win. \textbf{Add advantages and disadvantages.}

Another promising technique is the use of a Relational Markov Decision Process framework to generalize plans in different environments. \textbf{Add advantages and disadvantages.}
% Werking en positieve/negatieve punten toevoegen

Next, a paper is discussed which uses multiple AI techniques in a specific testing environment: MultiStar. Here, the performance of different reinforcement learning agents is tested against each other. The different techniques that are used are COMA and QMIX. \textbf{Add advantages and disadvantages.}
% Werking en positieve/negatieve punten toevoegen

Finally, AlphaStar is discussed, which is an AI system capable of beating professional players in a competitive StarCraft game. The neural network of AlphaStar applies an advanced architecture that is  capable of long-range sequence transduction with limited knowledge of the state. However, the training of this network relied on clear winning conditions between agents.

% Nog iets om het af  te maken?
At the end of this long research journey, it has become apparent that there are many different and unique algorithms which try to solve the distinct challenges presented by RTS games. Those that have been discussed in this review give a broad idea how creative the research community is and shows that AI is still very much an open-ended question.